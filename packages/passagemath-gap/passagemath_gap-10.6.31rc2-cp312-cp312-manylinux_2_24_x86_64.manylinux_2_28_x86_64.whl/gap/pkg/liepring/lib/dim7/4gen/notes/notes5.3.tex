
\documentclass[12pt]{article}
%%%%%%%%%%%%%%%%%%%%%%%%%%%%%%%%%%%%%%%%%%%%%%%%%%%%%%%%%%%%%%%%%%%%%%%%%%%%%%%%%%%%%%%%%%%%%%%%%%%%%%%%%%%%%%%%%%%%%%%%%%%%%%%%%%%%%%%%%%%%%%%%%%%%%%%%%%%%%%%%%%%%%%%%%%%%%%%%%%%%%%%%%%%%%%%%%%%%%%%%%%%%%%%%%%%%%%%%%%%%%%%%%%%%%%%%%%%%%%%%%%%%%%%%%%%%
\usepackage{amsfonts}
\usepackage{amssymb}
\usepackage{sw20elba}

%TCIDATA{OutputFilter=LATEX.DLL}
%TCIDATA{Version=5.50.0.2890}
%TCIDATA{<META NAME="SaveForMode" CONTENT="1">}
%TCIDATA{BibliographyScheme=Manual}
%TCIDATA{Created=Friday, July 26, 2013 12:33:18}
%TCIDATA{LastRevised=Friday, July 26, 2013 16:58:31}
%TCIDATA{<META NAME="GraphicsSave" CONTENT="32">}
%TCIDATA{<META NAME="DocumentShell" CONTENT="Articles\SW\mrvl">}
%TCIDATA{CSTFile=LaTeX article (bright).cst}

\newtheorem{theorem}{Theorem}
\newtheorem{axiom}[theorem]{Axiom}
\newtheorem{claim}[theorem]{Claim}
\newtheorem{conjecture}[theorem]{Conjecture}
\newtheorem{corollary}[theorem]{Corollary}
\newtheorem{definition}[theorem]{Definition}
\newtheorem{example}[theorem]{Example}
\newtheorem{exercise}[theorem]{Exercise}
\newtheorem{lemma}[theorem]{Lemma}
\newtheorem{notation}[theorem]{Notation}
\newtheorem{problem}[theorem]{Problem}
\newtheorem{proposition}[theorem]{Proposition}
\newtheorem{remark}[theorem]{Remark}
\newtheorem{solution}[theorem]{Solution}
\newtheorem{summary}[theorem]{Summary}
\newenvironment{proof}[1][Proof]{\noindent\textbf{#1.} }{{\hfill $\Box$ \\}}
\input{tcilatex}
\addtolength{\textheight}{30pt}

\begin{document}

\title{Algebra 5.3}
\author{Michael Vaughan-Lee}
\date{July 2013}
\maketitle

Algebra 5.3 has $p^{4}+5p^{3}+19p^{2}+64p+140+(p+6)\gcd (p-1,3)+(p+7)\gcd
(p-1,4)+\gcd (p-1,5)$ immediate descendants of order $p^{7}$ and $p$-class 3.

Algebra 5.3 has presentation 
\[
\langle a,b,c,d\,|\,ca,da,cb,db,dc,pa,pb,pc,pd,\,\text{class }2\rangle . 
\]%
So it has characteristic $p$ and derived algebra of order $p$ generated by $%
ba$, with all other commutators trivial. So if $L$ is an immediate
descendant of 5.3 then $L$ has class 3, $L_{3}$ is generated by $baa,bab$,
and the elements $ca,da,cb,db,dc,pa,pb,pc,pd$ are all linear combinations of 
$baa,bab$. The commutator structure of $L$ must correspond to one of the
algebras 7.21 -- 7.28 in the list of nilpotent Lie algebras over $\mathbb{Z}%
_{p}$ of order $p^{7}$. So we can assume that one of the following sets of
commutator relations holds. For any given set of commutator relations, $%
pa,pb,pc,pd$ are linear combinations of $baa,bab$. 
\begin{eqnarray*}
ca &=&cb=da=db=dc=0, \\
cb &=&da=db=dc=0,\,ca=bab, \\
cb &=&da=db=dc=0,\,ca=baa, \\
da &=&db=dc=0,\,ca=bab,\,cb=\omega baa, \\
ca &=&da=dc=0,\,cb=baa,\,db=bab, \\
da &=&dc=0,\,ca=db=bab,\,cb=baa, \\
da &=&dc=0,\,ca=db=bab,\,cb=\omega baa, \\
ca &=&cb=da=db=0,\,dc=baa, \\
cb &=&da=db=0,\,ca=bab,\,dc=baa.
\end{eqnarray*}

In 6 of these cases we are able to provide parametrized presentations with
fairly simple restrictions on the parameters, but in cases 4, 6 and 7 we
were unable to do this.

\section{Case 4}

We are able to provide parametrized presentations with fairly simple
restrictions on the parameters in Case 4 for, except for one presentation

\[
\langle a,b,c,d\,|\,ca-bab,\,cb-\omega baa,da,db,dc,\,pa-\lambda baa-\mu
bab,\,pb-\nu baa-\xi bab,\,pc,pd,\,\text{class }3\rangle . 
\]%
If we write the parameters $\lambda ,\mu ,\nu ,\xi $ in a matrix (which is
assumed to be non-singular)%
\[
A=\left( 
\begin{array}{cc}
\lambda & \mu \\ 
\nu & \xi%
\end{array}%
\right) , 
\]%
then two matrices give isomorphic algebras if and only if they are in the
same orbit under the action%
\[
A\rightarrow \frac{1}{\det P}PAP^{-1}, 
\]%
where $P$ lies in the group of non-singular matrices of the form%
\[
\left( 
\begin{array}{ll}
\alpha & \beta \\ 
\pm \omega \beta & \pm \alpha%
\end{array}%
\right) . 
\]%
This is the same action as appears in algebra 6.178 in the algebras of order 
$p^{6}$. In fact the subalgebra $\langle a,b,c\rangle $ here is 6.178. There
is a \textsc{Magma} program to compute the orbits in notes5.3.m. In the
notes on 6.178 I commented that it would be nice to do better than
complexity $p^{5}$ in the program to sort the orbits, and I see that the
program here has complexity $p^{4}$.

\section{Case 6}

In Case 6, $L$ satisfies the commutator relations $da=dc=0$, $ca=db=bab$, $%
cb=baa$. We write%
\[
\left( 
\begin{array}{c}
pa \\ 
pb \\ 
pc \\ 
pd%
\end{array}%
\right) =A\left( 
\begin{array}{c}
baa \\ 
bab%
\end{array}%
\right) 
\]%
where $A$ is $4\times 2$ matrix. Two matrices $A$ give isomorphic algebras
if they lie in the same orbit under the action%
\[
A\rightarrow \frac{1}{\alpha ^{2}+\beta ^{2}}\left( 
\begin{array}{cccc}
\alpha & -\beta & \gamma & \delta \\ 
\pm \beta & \pm \alpha & \pm \lambda & \pm \mu \\ 
0 & 0 & \alpha ^{2}-\beta ^{2} & -4\alpha \beta \\ 
0 & 0 & \pm \alpha \beta & \pm (\alpha ^{2}-\beta ^{2})%
\end{array}%
\right) A\left( 
\begin{array}{cc}
\pm \alpha & \mp \beta \\ 
\beta & \alpha%
\end{array}%
\right) ^{-1}. 
\]

There is a \textsc{Magma} program to compute the orbits in notes5.3.m.

\section{Case 7}

In Case 7, $L$ satisfies the commutator relations $da=dc=0$, $ca=db=bab$, $%
cb=\omega baa$. We write%
\[
\left( 
\begin{array}{c}
pa \\ 
pb \\ 
pc \\ 
pd%
\end{array}%
\right) =A\left( 
\begin{array}{c}
baa \\ 
bab%
\end{array}%
\right) 
\]%
where $A$ is $4\times 2$ matrix. Two matrices $A$ give isomorphic algebras
if they lie in the same orbit under the action%
\[
A\rightarrow \frac{1}{\alpha ^{2}+\omega \beta ^{2}}\left( 
\begin{array}{cccc}
\alpha & \beta & \gamma & \delta \\ 
\mp \omega \beta & \pm \alpha & \pm \lambda & \pm \mu \\ 
0 & 0 & \alpha ^{2}-\omega \beta ^{2} & 4\omega \alpha \beta \\ 
0 & 0 & \mp \alpha \beta & \pm (\alpha ^{2}-\omega \beta ^{2})%
\end{array}%
\right) A\left( 
\begin{array}{cc}
\pm \alpha & \pm \beta \\ 
-\omega \beta & \alpha%
\end{array}%
\right) ^{-1}. 
\]

There is a \textsc{Magma} program to compute the orbits in notes5.3.m.

\end{document}
