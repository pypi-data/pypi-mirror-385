%- if geometryFigure
\begin{figure}[!h]
	\centering
	\setlength{\fboxsep}{0pt}%
	\setlength{\fboxrule}{0.5pt}%
	\fbox{\includegraphics[width=0.90\textwidth]{\VAR{geometryUrl}}}%
	\caption{Plot della sezione trasversale}
\end{figure}
%- endif
\FloatBarrier

Si riportano nella tabella seguente le coordinate dei vertici della sezione ed il loro identificativo \emph{id}:

\begin{longtable}[c]{|c|>{\centering\arraybackslash}p{30mm}|>{\centering\arraybackslash}p{30mm}|}
	\caption{Tabella dei vertici \label{tab:vertices}} \\
	\hline
	\multirow{2}{*}{\textbf{id}} & $\boldsymbol{X}$            & $\boldsymbol{Y}$ \bigstrut \\ \cline{2-3}
	& \footnotesize{\textit{[mm]}} & \footnotesize{\textit{[mm]}} \\
	\endfirsthead
	
	\multicolumn{3}{c}{continuazione Tabella \ref{tab:vertices}}\\
	\hline
	\multirow{2}{*}{\textbf{id}} & $\boldsymbol{X}$            & $\boldsymbol{Y}$ \bigstrut \\ \cline{2-3}
	& \footnotesize{\textit{[mm]}} & \footnotesize{\textit{[mm]}} \\
	\endhead
	
	%- for v in vertices
	\hline
	\VAR{v.id} & \VAR{v.xPos} & \VAR{v.yPos} \\
	%- endfor
	\hline
\end{longtable}

%- if namedVertices
La sezione è di forma specifica \emph{\VAR{sectionShape}} con i seguenti parametri:

%- if rectangularShape
\begin{align*}
	b &= \VAR{ shape_width } \text{ \textit{mm}} & h &= \VAR{ shape_height } \text{ \textit{mm}}
\end{align*}
%- endif	

Le coordinate notevoli dei vertici sono espresse nella seguente tabella:

\begin{longtable}[c]{|c|>{\centering\arraybackslash}p{30mm}|>{\centering\arraybackslash}p{30mm}|}
	\caption{Tabella dei vertici notevoli \label{tab:namedVertices}} \\
	\hline
	\multirow{2}{*}{\textbf{id}} & $\boldsymbol{X}$            & $\boldsymbol{Y}$ \bigstrut \\ \cline{2-3}
	& \footnotesize{\textit{[mm]}} & \footnotesize{\textit{[mm]}} \\
	\endfirsthead
	
	\multicolumn{3}{c}{continuazione Tabella \ref{tab:namedVertices}}\\
	\hline
	\multirow{2}{*}{\textbf{id}} & $\boldsymbol{X}$            & $\boldsymbol{Y}$ \bigstrut \\ \cline{2-3}
	& \footnotesize{\textit{[mm]}} & \footnotesize{\textit{[mm]}} \\
	\endhead
	
	%- for n in namedVertices
	\hline
	\VAR{n.name} & \VAR{n.xPos} & \VAR{n.yPos} \\
	%- endfor
	\hline
\end{longtable}
%- endif

mentre nella seguente tabella si rappresenta la connettività dei triangoli:

\begin{longtable}[c]{|c|c|c|c|}
	\caption{Tabella dei triangoli e connettività \label{tab:triangles}} \\
	\hline
	\multirow{2}{*}{\textbf{id}}   & \multirow{2}{*}{\textbf{id\_1}}  &  \multirow{2}{*}{\textbf{id\_2}} &  \multirow{2}{*}{\textbf{id\_3}} \\ 
	                               &                                  &                                  &                                  \\
	\endfirsthead
	
	\multicolumn{4}{c}{continuazione Tabella \ref{tab:triangles}}\\
	\hline
    \multirow{2}{*}{\textbf{id}}   & \multirow{2}{*}{\textbf{id\_1}}  &  \multirow{2}{*}{\textbf{id\_2}} &  \multirow{2}{*}{\textbf{id\_3}} \\ 
                                   &                                  &                                  &                                  \\
	\endhead
	
	%- for t in triangles
	\hline
	\VAR{t.id} & \VAR{t.id1} & \VAR{t.id2} & \VAR{t.id3}\\
	%- endfor
	\hline
\end{longtable}

Di seguito la tabella relativa al posizionamento degli acciai, essendo \emph{id} l'identificativo della barra, $X$ ed $Y$ le coordinate, $d$ il diametro e \emph{idv} l'identificativo del vertice:

\begin{longtable}[c]{|c|>{\centering\arraybackslash}p{25mm}|>{\centering\arraybackslash}p{25mm}|>{\centering\arraybackslash}p{25mm}|c|}
	\caption{Tabella dei rinforzi \label{tab:rebars}} \\
	\hline
    \multirow{2}{*}{\textbf{id}} & $\boldsymbol{X}$             & $\boldsymbol{Y}$             & $\boldsymbol{d}$             & \multirow{2}{*}{\textbf{idv}} \\ \cline{2-4}
                                 & \footnotesize{\textit{[mm]}} & \footnotesize{\textit{[mm]}} & \footnotesize{\textit{[mm]}} &                               \\
	\endfirsthead
	
	\multicolumn{5}{c}{continuazione Tabella \ref{tab:rebars}}\\
	\hline
	\multirow{2}{*}{\textbf{id}} & $\boldsymbol{X}$             & $\boldsymbol{Y}$             & $\boldsymbol{d}$             & \multirow{2}{*}{\textbf{idv}} \\ \cline{2-4}
	                             & \footnotesize{\textit{[mm]}} & \footnotesize{\textit{[mm]}} & \footnotesize{\textit{[mm]}} &                               \\
	\endhead
	
	%- for r in rebars
	\hline
	\VAR{r.id} & \VAR{r.xPos} & \VAR{r.yPos} & \VAR{r.diam} & \VAR{r.id_v} \\
	%- endfor
		
	\hline
\end{longtable}




