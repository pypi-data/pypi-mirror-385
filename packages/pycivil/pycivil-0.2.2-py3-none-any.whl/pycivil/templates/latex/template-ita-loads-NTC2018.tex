Nella seguente tabella sono elencate le caratteristiche di sollecitazione nelle
varie combinazioni di carico.

\begin{longtable}[c]{|c|r|r|r|p{3.5cm}|c|c|}
    \caption{Combinazioni di carico e sollecitazioni \label{long}}\\
    \hline
    \multirow{4}{*}{\textbf{id}} & \mcsym{F_x}  & \mcsym{F_y}  & \mcsym{F_z}  & \multirow{4}{*}{\textbf{Descrizione}} & \multirow{4}{*}{\textbf{S.L.}} & \multirow{4}{*}{\textbf{FR.}} \bigstrut \\\cline{2-4}
                                 &      \multicolumn{3}{ c|}{\footnotesize{\emph{[KN]}}}           &                                       &                                &                               \bigstrut \\\cline{2-4}
                                 & \mcsym{M_x}  & \mcsym{M_y}  & \mcsym{M_z}  &                                       &                                &                               \bigstrut \\\cline{2-4}
                                 &      \multicolumn{3}{ c|}{\footnotesize{\emph{[KNm]}}}          &                                       &                                & \\ \hline
    \endfirsthead

    \multicolumn{7}{ c }{continuazione della Tabella \ref{long}}\\
    \hline
    \multirow{4}{*}{\textbf{id}} & $\boldsymbol{F_x}$  & $\boldsymbol{F_y}$  & $\boldsymbol{F_z}$  & \multirow{4}{*}{\textbf{Descrizione}} & \multirow{4}{*}{\textbf{S.L.}} & \multirow{4}{*}{\textbf{FR.}} \bigstrut \\\cline{2-4}
                                 &      \multicolumn{3}{ c|}{\footnotesize{\emph{[KN]}}}           &                                       &                                &                               \bigstrut \\\cline{2-4}
                                 & $\boldsymbol{M_x}$  & $\boldsymbol{M_y}$  & $\boldsymbol{M_z}$  &                                       &                                &                               \bigstrut \\\cline{2-4}
                                 &      \multicolumn{3}{ c|}{\footnotesize{\emph{[KNm]}}}          &                                       &                                & \\ \hline
    \endhead

    %\hline
    %\multicolumn{7}{ c }{continua \ldots}\\
    %\endfoot

    %- for force in forces
    \multirow{2}{*}{ \VAR{force.id} } & \VAR{force.Fx} & \VAR{force.Fy} & \VAR{force.Fz} & \multirow{2}{*}{ \VAR{force.descr} } & \multirow{2}{*}{ \VAR{force.limitState} } & \multirow{2}{*}{ \VAR{force.frequency} } \bigstrut \\\cline{2-4}
                                      & \VAR{force.Mx} & \VAR{force.My} & \VAR{force.Mz} & & & \\ \hline
    %- endfor
\end{longtable}

Ogni combinazione ha un identificativo unico (\textbf{id}).

La colonna \textbf{S.L.} contiene il tipo di stato limite rappresentato dalle
seguenti possibilità:

\begin{description}
    \item[SLU] Stati limite ultimi
    \item[SLE] Stati limite di esercizio
\end{description}

La colonna \textbf{FR.} contiene la combinazione di carico che rappresenta la frequenza:

\begin{description}
    \item[RAR] Combinazione di carico rara
    \item[FRE] Combinazione di carico frequente
    \item[QP] Combinazione di carico quasi permanente
\end{description}
