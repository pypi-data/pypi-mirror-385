\documentclass{article}
% common styling and macros shared by all proof files

\usepackage[top=1in, right=1in, left=1in, bottom=1.5in]{geometry}

\usepackage{amsmath,amsthm,amsfonts,amssymb,amscd}
\usepackage{listings}
\usepackage{hyperref}
\usepackage{xcolor}
\usepackage{xr}

\usepackage{enumerate} 
\usepackage{physics}
\usepackage{fancyhdr}
\usepackage{hyperref}
\usepackage{graphicx}
\usepackage{tcolorbox}
\usepackage{catchfile}
\usepackage{pdftexcmds}
\usepackage[T1]{fontenc}

% hyperref
\hypersetup{
  colorlinks=true,
  linkcolor=blue,
  linkbordercolor={0 0 1}
}

% \contrib macro to indicate inclusion in "contrib".
\usepackage{tcolorbox}
\newtcolorbox{warn_box}{colback=red!5!white,colframe=red!75!black}
\newcommand{\contrib}{{\begin{warn_box}This proof resides in \textbf{``contrib''} because it has not completed the vetting process.\end{warn_box}}} 
\newcommand{\floatingPoint}{{\begin{warn_box}This implementation is susceptible to floating-point vulnerabilities.\end{warn_box}}} 

% asOfCommit macro to version a code dependency. Arguments:
%    #1: relative path to file you are dependent on
%    #2: commit hash it was last edited. If outdated, this should be the second hash in the footnoote. Otherwise,
%            git log -n 1 --pretty=format:%h -- path/to/file.rs
\makeatletter
\ifnum\pdf@shellescape=1
   % "private" command that builds a link to a blob
  \newcommand{\linkOpendpBlob}[3]{%
    \href{https://github.com/opendp/opendp/blob/#1/#2#3}{\path{#3} at commit #1}}

  % latex macro expansion has a separate phase for \input evaluation
  %     immediately evaluate a command to write a temp file to ./out containing the current directory
  \immediate\write18{[ ! -f out/cwd.txt ] && (mkdir -p out && git rev-parse --show-prefix | sed "s|_|\@backslashchar\@backslashchar\@backslashchar_|g" > out/cwd.txt)}
  %     ...and then retrieve the current working directory by loading the temp file
  \CatchFileDef\GitWorkingDir{out/cwd.txt}{\endlinechar=-1}

  % command for building the (up to date) or (outdated) status
  \newcommand{\fileStatus}[2]{%
  \setbox0=\hbox{\input|"git --no-pager log -n1 --pretty='\@percentchar H' #1 | grep -E '^#2.*'"\unskip}\ifdim\wd0=0pt
        (outdated\footnote{See new changes with \texttt{git diff #2..\input|"git --no-pager log -n1 --pretty='\@percentchar h' #1" \GitWorkingDir\path{#1}}})\else
        (up to date)\fi
  }

  \newcommand{\asOfCommit}[2]{%
      % permalink the target
      \linkOpendpBlob{#2}{\GitWorkingDir}{#1}
      % conditionally add (outdated) or (up to date) depending on matching commit hash
      \fileStatus{#1}{#2}%
  }
\else
  % simplified command if shell-escape not enabled
  \newcommand{\asOfCommit}[2]{#1 at commit #2 (unknown status\footnote{Shell-escape is not enabled. Enable \texttt{--shell-escape} to check if this proof is up-to-date with the code.})}
\fi
\makeatother

% \vettingPR macro to link a PR. Arguments:
%    #1: PR number
\newcommand{\vettingPR}[1]{\href{https://github.com/opendp/opendp/pull/#1}{Pull Request \##1}}

% for links to rustdoc items in OpenDP. Arguments:
%    #1: path to item, and designation as trait, struct, fn, etc.
%    #2: item name
\makeatletter
\ifnum\pdf@shellescape=1
  % latex macro expansion has a separate phase for \input evaluation
  %     immediately evaluate a command to write a temp file to ./out containing the base path
  \immediate\write18{[ ! -f out/rustdoc.txt ] && mkdir -p out && ([ -z `kpsewhich --var-value OPENDP_RUSTDOC_PORT` ] && echo "https://docs.rs/opendp/`head -n 1 \@backslashchar`git rev-parse --show-toplevel\@backslashchar`/VERSION | sed 's|.*-dev.*|latest|g'`" || echo "http://localhost:`kpsewhich --var-value OPENDP_RUSTDOC_PORT`") > out/rustdoc.txt}
  %     ...and then retrieve the base path by loading the temp file
  \CatchFileDef\OpenDPRustdocBase{out/rustdoc.txt}{\endlinechar=-1}
\else
  % if shell commands are not enabled, just claim latest
  \newcommand{\OpenDPRustdocBase}{https://docs.rs/opendp/latest}
\fi
\makeatother
\newcommand{\rustdoc}[2]{\href{\OpenDPRustdocBase/opendp/#1.#2.html}{\texttt{#2}}}

% for links to external dependencies. Arguments:
%    #1: crate name
%    #2: path to item, and designation as trait, struct, fn, etc.
%    #3: item name
\newcommand{\docsrs}[3]{\href{https://docs.rs/#1/latest/#1/#2.#3.html}{\texttt{#3}}}

% minted (pseudocode)
\definecolor{codegreen}{rgb}{0,0.6,0}
\definecolor{codegray}{rgb}{0.5,0.5,0.5}
\definecolor{codepurple}{rgb}{0.58,0,0.82}
\definecolor{backcolour}{rgb}{0.95,0.95,0.92}

\lstdefinestyle{mystyle}{
    backgroundcolor=\color{backcolour},   
    commentstyle=\color{codegreen},
    keywordstyle=\color{magenta},
    numberstyle=\tiny\color{codegray},
    stringstyle=\color{codepurple},
    basicstyle=\ttfamily\footnotesize,
    breakatwhitespace=false,         
    breaklines=true,                 
    captionpos=b,                    
    keepspaces=true,                 
    numbers=left,                    
    numbersep=5pt,                  
    showspaces=false,                
    showstringspaces=false,
    showtabs=false,                  
    tabsize=2
}

\lstset{style=mystyle}

% common commands
\theoremstyle{definition}
\newtheorem{theorem}{Theorem}[section]
\newtheorem{lemma}[theorem]{Lemma}
\newtheorem{definition}[theorem]{Definition}
\newtheorem{warning}{Warning}
\newtheorem{corollary}{Corollary}
\newtheorem{proposition}{Proposition}
\newtheorem{remark}{Remark}
\newtheorem{observation}{Observation}
\newtheorem{note}{Note}

\newcommand{\vicki}[1]{{ {\color{olive}{(vicki)~#1}}}}
\newcommand{\hanwen}[1]{{ {\color{purple}{(hanwen)~#1}}}}
\newcommand{\zach}[1]{{ {\color{red}{(zach)~#1}}}}

\newcommand{\MultiSet}{\mathrm{MultiSet}}
\newcommand{\len}{\mathrm{len}}
\newcommand{\din}{\texttt{d\_in}}
\newcommand{\dout}{\texttt{d\_out}}
\newcommand{\T}{\texttt{T} }
\newcommand{\F}{\texttt{F} }
\newcommand{\Map}{\texttt{Map}}
\newcommand{\X}{\mathcal{X}}
\newcommand{\Y}{\mathcal{Y}}
\newcommand{\True}{\texttt{True}}
\newcommand{\False}{\texttt{False}}
\newcommand{\clamp}{\texttt{clamp}}
\newcommand{\function}{\texttt{function}}
\newcommand{\float}{\texttt{float }}
\newcommand{\questionc}[1]{\textcolor{red}{\textbf{Question:} #1}}


\newcommand{\validTransformation}[2]{%
  \begin{theorem}
  For every setting of the input parameters #1 to #2 such that the given preconditions
  hold, #2 raises an error (at compile time or run time) or returns a valid transformation. A valid transformation has the following properties:
  \begin{enumerate}
      \item \textup{(Data-independent runtime errors).}
      For every pair of members $x$ and $x'$ in \texttt{input\_domain},
      $\texttt{invoke}(x)$ and $\texttt{invoke}(x')$ either both return the same error or neither return an error. 

      \item \textup{(Appropriate output domain).} 
      For every member $x$ in \texttt{input\_domain}, $\function(x)$ is in \texttt{output\_domain} or raises a data-independent runtime error.
      
      \item \textup{(Stability guarantee).} 
      For every pair of members $x$ and $x'$ in \texttt{input\_domain} and for every pair $(\din, \dout)$, 
      where \din\ has the associated type for \texttt{input\_metric} and \dout\ has the associated type for \\ \texttt{output\_metric}, 
      if $x, x'$ are \din-close under \texttt{input\_metric}, $\texttt{stability\_map}(\din)$ does not raise an error,
      and $\texttt{stability\_map}(\din) = \dout$, 
      then $\function(x), \function(x')$ are $\dout$-close under \texttt{output\_metric}.
  \end{enumerate}
  \end{theorem}
}


\newcommand{\validMeasurement}[2]{%
  For every setting of the input parameters #1 to #2 such that the given preconditions
  hold, #2 raises an error (at compile time or run time) or returns a valid measurement. A valid measurement has the following properties:
  \begin{enumerate}
      \item \textup{(Data-independent runtime errors).}
      For every pair of members $x$ and $x'$ in \texttt{input\_domain},
      $\texttt{invoke}(x)$ and $\texttt{invoke}(x')$ either both return the same error or neither return an error. 

      \item \textup{(Privacy guarantee).}
      For every pair of members $x$ and $x'$ in \texttt{input\_domain} and for every pair $(\din, \dout)$,
      where \din\ has the associated type for \texttt{input\_metric} and \dout\ has the associated type for \\ \texttt{output\_measure},
      if $x, x'$ are \din-close under \texttt{input\_metric}, $\texttt{privacy\_map}(\din)$ does not raise an error,
      and $\texttt{privacy\_map}(\din) = \dout$,
      then $\function(x), \function(x')$ are $\dout$-close under \texttt{output\_measure}.
  \end{enumerate}
}

\newcommand{\validPostprocessor}[2]{%
  For every setting of the input parameters #1 to #2 such that the given preconditions
  hold, #2 raises an error (at compile time or run time) or returns a valid postprocessor. A valid postprocessor has the following property:
  \begin{enumerate}
      \item \textup{(Data-independent errors).}
      For every pair of members $x$ and $x'$ in \texttt{input\_domain},
      $\function(x), \function(x')$ either both raise the same error, 
      or neither raise an error.
  \end{enumerate}
}

\newcommand{\validOdometer}[2]{%
  For every setting of the input parameters #1 to #2 such that the given preconditions
  hold, #2 raises an error (at compile time or run time) or returns a valid odometer. 
  A valid odometer has the following properties:
  \begin{enumerate}
      \item \textup{(Data-independent runtime errors).}
      For every pair of members $x$ and $x'$ in \texttt{input\_domain},
      $\texttt{invoke}(x)$ and $\texttt{invoke}(x')$ either both return the same error or neither return an error. 

      \item \textup{(Valid odometer queryable).}
      For every member $x$ in \texttt{input\_domain},
      where $\function(x)$ does not raise an error,
      $\function(x)$ returns a valid odometer queryable.
  \end{enumerate}
}

\newcommand{\validOdometerQueryable}[2]{%
  For every setting of the input parameters #1 to #2 such that the given preconditions
  hold, #2 raises a data-independent error or returns a valid odometer queryable (\texttt{queryable}). 
  A valid odometer queryable is an \rustdoc{core/type}{OdometerQueryable} with the following properties:
  \begin{enumerate}
    \item \textup{(Data-independent errors).}
    For every pair of members $x$ and $x'$ in \texttt{input\_domain}, and every query $q$,
    $\texttt{queryable.invoke}_x(q)$ and $\texttt{queryable.invoke}_{x'}(q)$ either both return the same error or neither return an error. 

    \item \textup{(Privacy Guarantee).}
    For every pair of members $x$ and $x'$ in \texttt{input\_domain}, 
    every adversary $\mathcal{A}$, and for every pair $(\din, \dout)$, 
    where \din\ has the associated type for \texttt{input\_metric} and \dout\ has the associated type for \texttt{output\_measure}, 
    if $x$ and $x'$ are \din-close under \texttt{input\_metric}, $\texttt{queryable.eval(privacy\_loss)}$ does not return an error, 
    and $\texttt{queryable.eval(privacy\_loss)} = \dout$, 
    then $\texttt{View}(\mathcal{A} \leftrightarrow \texttt{queryable}_x), \texttt{View}(\mathcal{A} \leftrightarrow \texttt{queryable}_{x'})$ are $\dout$-close under \texttt{output\_measure},
    where $\texttt{queryable}_x$ denotes all messages received by $\mathcal{A}$ under $x$.
  \end{enumerate}
}


\title{\texttt{fn make\_private\_group\_by}}
\author{Michael Shoemate}
\date{}

\begin{document}

\maketitle

\contrib
Proves soundness of \texttt{fn make\_private\_group\_by} in \asOfCommit{mod.rs}{0db9c6036}.

\section{Hoare Triple}
\subsection*{Precondition}
\subsubsection*{Compiler-verified}
\begin{itemize}
    \item Generic \texttt{MI} must implement trait \texttt{UnboundedMetric}.
    \item Generic \texttt{MO} must implement trait \texttt{ApproximateMeasure}.
\end{itemize}

\subsubsection*{User-verified}
None

\subsection*{Pseudocode}
\lstinputlisting[language=Python,firstline=2,escapechar=|]{./pseudocode/make_private_group_by.py}

\subsubsection*{Postconditions}
\begin{theorem}
    \label{postcondition}
    \validMeasurement{\texttt{(input\_domain, input\_metric, \\output\_measure, plan, global\_scale, threshold, MI, MO)}}{\texttt{make\_private\_group\_by}}
\end{theorem}
\section{Proof}

We now prove the postcondition (Theorem~\ref{postcondition}).
\begin{proof}

The function logic breaks down into parts:
\begin{enumerate}
    \item establish stability of group by (line \ref{line:input_stability})
    \item prepare for release of \texttt{aggs} (line \ref{line:prep-release-aggs})
    \item prepare for release of \texttt{keys} (line \ref{line:prep-release-keys})
    \begin{enumerate}
        \item reconcile information about the threshold (line \ref{line:reconcile-threshold})
        \item update key sanitizer (line \ref{line:final-sanitizer})
    \end{enumerate}
    \item build final measurement (line \ref{line:build-meas})
    \begin{enumerate}
        \item construct function (line \ref{line:function})
        \item construct privacy map (line \ref{line:privacy-map})
    \end{enumerate}
\end{enumerate}

\rustdoc{measurements/make\_private\_lazyframe/group\_by/matching/fn}{match\_group\_by} on line \ref{line:match-group-by} returns 
\texttt{input} (the input plan), \texttt{group\_by} (the grouping keys), \texttt{aggs} (the list of expressions to compute per-partition),
and \texttt{key\_sanitizer} (details on how to sanitize the key-set).

\subsection{Stability of grouping}
By the postcondition of \texttt{StableDslPlan.make\_stable}, \texttt{t\_prior} is a valid transformation (line \ref{line:tprior}).

The loop on line \ref{line:group-by-stability} ensures that each column in \texttt{group\_by} is stable,
and that the encoding of data in each group-by column is not data-dependent.
Therefore data is grouped in a stable manner, with no data-dependent encoding or exceptions.

\subsection{Prepare to release \texttt{aggs}}
\texttt{margin} denotes what is considered public information about the key set, 
pulled from descriptors in the input domain (line \ref{line:groupcols}).
An upper bound on the total number of groups can be statically derived via the length of the public keys in the join.
Line \ref{line:keys-len} retrieves this information from the public keys and assigns it to the margin.
\texttt{is\_join} indicates that key sanitization will occur via a join.

Line \ref{line:joint} starts constructing a joint measurement 
for releasing the per-partition aggregations \texttt{aggs}.
Each measurement's input domain is the wildcard expression domain,
used to prepare computations that will be applied over data grouped by \texttt{group\_by}.

By the postcondition of \texttt{make\_basic\_composition}, 
\texttt{m\_exprs} is a valid measurement that prepares a batch of expressions that, 
when executed via \texttt{f\_comp}, satisfies the privacy guarantee of \texttt{f\_privacy\_map}.

Now that we've prepared the necessary prerequisites for privatizing the aggregations,
we switch to privatizing the keys.

\subsection{Prepare to release \texttt{keys}}
\texttt{key\_sanitization} needs to be updated with information that was not available in the initial match on line \ref{line:match-group-by}.
\begin{itemize}
    \item When joining, we need expressions for filling null values corresponding to partitions that don't exist in the sensitive data.
    \item When filtering, a threshold may be passed into the constructor, and we must determine a suitable column to filter/threshold against.
\end{itemize}

By the definition of \texttt{m\_aggs}, invokation returns a list of expressions and fill expressions.
These will be used for the filtering sanitization and join sanitization, respectively.

\subsubsection{Reconcile information when filtering}
Line \ref{line:reconcile-threshold} reconciles the threshold information.
\begin{itemize}
    \item In the setting where grouping keys are considered public, 
    or key sanitization is handled via a join, no thresholding is necessary (line \ref{line:no-thresholding}).
    \item Otherwise, if the key sanitizer contains filtering criteria (line \ref{line:sanitizer-threshold}),
    then by the postcondition of \texttt{find\_len\_expr}, filtering on \texttt{name} can be used to satisfy 
    $\delta$-approximate DP.
    \texttt{noise} of type \rustdoc{measurements/make\_private\_expr/fn}{NoisePlugin} details the noise distribution and scale. 
    \texttt{threshold\_info} then contains the column name, noise distribution, threshold value and whether a filter needs to be inserted into the query plan.
    In this case, since the threshold comes from the query plan, it is not necessary to add it to the query plan, and is therefore false.
    \item In the case that a threshold has been provided to the constructor (line \ref{line:constructor-threshold}),
    then \texttt{find\_len\_expr} will search for a suitable column to threshold on, 
    returning with the \texttt{name} and \texttt{noise} distribution of the column.
    Since the threshold comes from the constructor and not the plan, it will be necessary to add this filtering threshold to the query plan (explaining the true value).
    \item By line \ref{line:not-private} no suitable filtering criteria have been found, 
    and by the first case there is no suitable invariant for the margin or explicit join keys,
    so it is not possible to release the keys in a way that satisfies differential privacy,
    and the constructor refuses to build a measurement.
\end{itemize}

In common use through the context API, if a mechanism is allotted a delta parameter for stable key release but doesn't already satisfy approximate-DP,
then a search is conducted for the smallest suitable threshold parameter.
The branching logic from line \ref{line:reconcile-threshold} is intentionally written to ignore the constructor threshold 
when a suitable filtering threshold is already detected in the plan, to avoid overwriting/changing it.

\subsubsection{Update key sanitizer}
We now update \texttt{key\_sanitization} starting from line \ref{line:final-sanitizer}:
\begin{itemize}
    \item When filtering (line \ref{line:incorporate-threshold}), \texttt{threshold\_info} will always be set. 
    \texttt{threshold\_expr} reflects the reconciled criteria, using the chosen filtering column and threshold. 
    This threshold expression is applied either way the logic branches on line \ref{line:new-threshold-sanitizer}.
    The first case preserves any additional filtering criteria that was already present in the plan, but not used for key release.
    \item When joining (line \ref{line:incorporate-join-fill}) the sanitizer needs a way to fill missing values from partitions missing in the data.
    This is provided by \texttt{null\_exprs}, which contain imputation strategies for filling in missing values 
    in a way that is indistinguishable from running the mechanism on an empty partition.
\end{itemize}

\texttt{key\_sanitizer} now contains all necessary information to ensure that the keys are sanitized, 
and will be used to construct the function.
\texttt{threshold\_info} and \texttt{is\_join} are consistent with \texttt{key\_sanitizer}, 
and will be used to construct the privacy map.

\subsection{Build final measurement}
\subsubsection{Function}
Line \ref{line:function} builds the function of the measurement,
using all of the properties proven of the variables established thus far.
The function returns a \texttt{DslPlan} that applies each expression from \texttt{m\_exprs}
to \texttt{arg} grouped by \texttt{keys}.
\texttt{key\_sanitizer} is conveyed into the plan, if set, 
to ensure that the keys are also privatized if necessary.

In the case of the join privatization, by the definition of \texttt{KeySanitizer},
the join will either be a left or right join.
The branching swaps the input plan and labels plan to ensure that the sensitive input data is always joined against the labels,
but using the same join type as in the original plan.
Once the join is applied, the fill imputation expressions are applied, hiding which partitions don't exist in the original data.

It is assumed that the emitted DSL is executed in the same fashion as is done by Polars.
This proof/implementation does not take into consideration side-channels involved in the execution of the DSL.

\subsubsection{Privacy Map}
Line \ref{line:privacy-map} builds the privacy map of the measurement.
The measurement for each expression expects data set distances in terms of a triple:
\begin{itemize}
    \item $L^0$: the greatest number of groups that can be influenced by any one individual. 
    This is bounded above by \texttt{bound.num\_groups} and more loosely by \texttt{margin.max\_groups},
    but can also be bounded by the $L^1$ distance on line \ref{line:l0-from-l1}.
    \item $L^\infty$: the greatest number of records that can be added or removed by any one individual in each partition. 
    This is bounded above by \texttt{bound.per\_group} and more loosely by \texttt{margin.max\_length},
    but can also be bounded by the $L^1$ distance on line \ref{line:li-from-l1}.
    \item $L^1$: the greatest total number of records that can be added or removed across all partitions.
    This is bounded by per-group contributions when all data is in one group on line \ref{line:trivial-l1},
    but can also be bounded by the product of the $L^0$ and $L^\infty$ bounds on line \ref{line:l1-from-l0-li}.
\end{itemize}

By the postcondition of \texttt{f\_privacy\_map}, the privacy loss of releasing the output of \texttt{aggs}, 
when grouped data sets may differ by this distance triple,
is \texttt{d\_out}.

We also need to consider the privacy loss from releasing \texttt{keys}.
On line \ref{line:privacy-map-static-keys} under the \texttt{public\_info} invariant, or under the join sanitization, 
releases on any neighboring datasets $x$ and $x'$ will share the same key-set,
resulting in zero privacy loss.

We now adapt the proof from \cite{rogers2023unifyingprivacyanalysisframework} (Theorem 7) 
to consider the case of stable key release from line \ref{line:privacy-map-threshold}.
Consider $S$ to be the set of labels that are common between $x$ and $x'$.
Define event $E$ to be any potential outcome of the mechanism for which all labels are in $S$
(where only stable partitions are released).
We then lower bound the probability of the mechanism returning an event $E$.
In the following, $c_j$ denotes the exact count for partition $j$,
and $Z_j$ is a random variable distributed according to the distribution used to release a noisy count.

\begin{align*}
    \Pr[E] &= \prod_{j \in x \backslash x'} \Pr[c_j + Z_j \le T] \\
    &\ge \prod_{j \in x \backslash x'} \Pr[\Delta_\infty + Z_j \le T] \\
    &\ge \Pr[\Delta_\infty + Z_j \le T]^{\Delta_0}
\end{align*}

The probability of returning a set of stable partitions ($\Pr[E]$) 
is the probability of not returning any of the unstable partitions.
We now solve for the choice of threshold $T$ such that $\Pr[E] \ge 1 - \delta$.

\begin{align*}
    \Pr[\Delta_\infty + Z_j \le T]^{\Delta_0} &= \Pr[Z_j \le T - \Delta_\infty]^{\Delta_0} \\
    &= (1 - \Pr[Z_j > T - \Delta_\infty])^{\Delta_0}
\end{align*}

Let \texttt{d\_instability} denote the distance to instability of $T - \Delta_\infty$.
By the postcondition of \\ \texttt{integrate\_discrete\_noise\_tail},
the probability that a random noise sample exceeds \texttt{d\_instability} is at most \texttt{delta\_single}.
Therefore $\delta = 1 - (1 - \texttt{delta\_single})^{\Delta_0}$.
This gives a probabilistic-DP or probabilistic-zCDP guarantee,
which implies approximate-DP or approximate-zCDP guarantees respectively.
This privacy loss is then added to \texttt{d\_out}.

Together with the potential increase in delta for the release of the key set,
then it is shown that \function(x), \function(x') are \dout-close under \texttt{output\_measure}.

\end{proof}

\bibliographystyle{alpha}
\bibliography{ref}
\end{document}